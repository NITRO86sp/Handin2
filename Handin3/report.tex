\documentclass{tufte-handout}
\usepackage{amsmath}
\usepackage[utf8]{inputenc}
\usepackage{mathpazo}
\usepackage{booktabs}
\usepackage{microtype}

\pagestyle{empty}


\title{Closest Pair Report}
\author{Mark Dear, Stelios Papaoikonomou, Radek Niemczyk and Rasmus Løbner Christensen}

\begin{document}
  \maketitle

  \section{Results}

  Our implementation produces the expected results on all input files. Comparing with the given output file, we get the same values for all closest pair of points, on each input file.

  \section{Implementation details}

  The pseudo-code states the following (p. 136 of Kleinberg and Tardos, \emph{Algorithm Design}, Addison--Wesley 2008.):

a) 
Construct Px and Py in O(n log n) time

b) 
Construct Qx, Qy, Rx, Ry in O(n) time 

c)
Construct Sy in O(n) time

d)
For each point in Sy compute distance from 'delta' \newline
\bigskip
While comparing our implementation with the complexity of the actual pseudo algorithm, we will focus on the exact same points 
(though we have looked through our methods, e.g. to check for unnecessary loops etc.) \linebreak
Therefore, we will repeat the points from above, and hereafter argue for the complexity of our own implementation.

\bigskip
\emph{a) Construct Px and Py in O(n log n) time} \linebreak
The class, Parser, does this. It takes the output from the FileReader class (which reads a file via the Java File library), and iterate through each
line in the given file. The iteration is done via a for loop. For each line we use the trim() method, and then checks whether the trimmed string
begins with either EOF or a digit. If the line begins with EOF, we stop looking at the lines. Otherwise, if the trimmed string contains a digit,
we split this line on widespace (one or more widespaces). Hereafter parsing the values as doubles, and add these to Px and Py.
Lastly we sort Px and Py by x/y-coordinate, by using java's Comparator class.
We assume all the above (except the sorting), to run in constant time, hence we get the complexity of: 
O(k + (n log n)) => O(n log n) due to Comparator's complexity of O(n log n). \newline
\pagebreak

\emph{b) Construct Qx, Qy, Rx, Ry in O(n) time } \linebreak
Our implementation uses Java's built in List.Sublist method. Looking at the documentation this is guaranteed to run in O(1) for ArrayLists. 
The method doing the splitting of lists, is called 'constructQR'. But since we assume that the worst case require us to split n lists,
we get the run-time of O(n).

\bigskip
\emph{c) Construct Sy in O(n) time} \linebreak
The current implementation runs through Px (which is of size n), and checks whether each Point lies within 'delta' of L. During this it 
creates a list. The running time of this operation should be O(n). 

\bigskip
\emph{d) For each point in Sy compute distance from 'delta'} \linebreak
In order to do this, our implementation goes through each element in Sy (which worst-case is of size n), 
and then checks for the 15 (if they exists) next points in Sy, whether these constitute a smaller pair with the current element. 
The complexity is therefore O(15n), since we assume the for loop over 15 elements to be a constant, which is the same as O(n).

\bigskip
Total runtime is therefore: O(n log n) + the initial sorting which takes O(n log n).

\end{document}
